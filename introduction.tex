\section{Introduction}

Since the birth of Bitcoin in 2008\cite{nakamoto2008}, many other cryptocurrencies have been introduced. 
It is without a doubt that this flourishing ecosystem has evolved into an enormous financial market. 
Cryptocurrencies are traded against fiat (e.g. USD, AUD, EUR) or against each other. 
However, due to the lack of interoperability between different blockchains, most of the trades are executed on centralized exchanges. 
Due to regulations, these centralized exchanges have integrated complex KYC (Know Your Customer) procedures where traders have to go through lengthy processes to prove their identity. 
In addition, traders give up control over their hard-earned coins by depositing them in the exchange so that they can execute trades.
The trader now has to trust the exchange to manage their funds according to the highest standards, to protect them against thieves or not lose them otherwise. 
This trust was misused more than once in the past and billions of dollars in user funds have been lost\cite{exchangeHacks}. 

One could say that these centralized exchanges are now a relic of the past. 
A new era of decentralized exchanges has started, adhering to the core idea of Bitcoin: censorship resistance at all levels. 

Decentralized exchanges powered by atomic swaps, first introduced in 2015 by TierNolan\cite{TierNolan2013}, can now promise more guarantees in terms of security and privacy to traders. 

The original idea of atomic swaps uses HTLCs (Hash Time-Lock Contracts), imposing certain requirements on the underlying blockchains: (1) they must support scripts so that one can build hash locks; and (2) they must support timelocks. 

Technology has evolved and, with advances in cryptography, a new way of cross-chain atomic swaps using adaptor signatures is gaining traction.  

Atomic swaps using adaptor signatures (also referred to as Scriptless Scripts) have several advantages over traditional atomic swaps using HTLCs: 
(1) contrary to HTLCs where the same hash has to be used on each chain, transactions involved in an atomic swap using adaptor signatures cannot be linked; and 
(2) since no script is involved, the on-chain footprint is reduced which makes the atomic swap cheaper.

Within this work we present our current efforts on cross-chain atomic swaps using adaptor signatures. 
In particular, we show how adaptor signatures can be employed to swap between Monero and Bitcoin. Notably, the former does not support scripts or timelocks. 