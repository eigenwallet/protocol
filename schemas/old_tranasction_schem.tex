% Define block styles

%arrow style
\tikzstyle{line} = [draw, -latex']

%lable style
\tikzstyle{lables} = [midway, above, sloped]

% our custom styles

\tikzstyle{transaction} = 		[rectangle, text width=5em, text centered, rounded corners, minimum height=4.5em, minimum width=6em]

\tikzstyle{txLabel} = 			[rectangle, text width=5em, text centered, rounded corners, minimum height=2em, minimum width=6em]

%If condition
\tikzstyle{IF} = 				[diamond, draw, fill=gray!20]


\tikzstyle{txSpendingLabel} = 	[rectangle, text width=5em, font=\scriptsize, text centered, rounded corners]

% Sticky note style
\tikzstyle{stickynote} = [rectangle, font=\tiny, text centered, rounded corners=2pt, fill=yellow!80, draw=orange!50, inner sep=3pt, drop shadow={opacity=0.3, shadow xshift=1pt, shadow yshift=-1pt}]


\newcommand\TxBox[5]{%
 	  	\node[rectangle] at ({#1},{#2}) (center) {};
   
   
   		\node[txLabel,above=1em of center] (nodeLabel) {#3};

		% spending label
   		\node[txSpendingLabel, below=1em of center] (spendingLabel) {#4};
   
	 	%spending condition like a if condition  
	   	\node[{#5}, right=1.8em of spendingLabel] (spendingCondition){};
   
	   	\node[rectangle, draw, minimum height=1.5em, minimum width=6.3em, below=1em of center] (labelBox) {};
   
   		% box around the whole tx
	   	\node[rectangle, draw, rounded corners, minimum height=5em, minimum width=7em, below=-0.em of center] (txBox) {};
}


\newcommand\TxBoxWithoutCondition[4]{%
 \TxBox{#1}{#2}{#3}{#4}{}
}

\newcommand\TxBoxWithCondition[4]{%
 \TxBox{#1}{#2}{#3}{#4}{IF}
}

% Generic sticky note command
% Usage: \StickyNote[width]{targetNode}{position}{uniqueLabel}{text}
% Example: \StickyNote{txBox}{below=3em}{note1}{Alice locks Bitcoin}
% Example with width: \StickyNote[10em]{txBox}{below=3em}{note1}{Alice locks Bitcoin}
% Position examples: below=3em, right=2em, above left=1em and 2em
% Default width: 6em
\newcommand\StickyNote[5][6em]{%
	\node[stickynote, text width=#1, #3 of #2, scale=0.7] (#4) {#5};
	\draw[->, orange!70, thick] (#4) -- (#2);
}


\begin{figure}[htb]
	\centering
	\begin{tikzpicture}
		[-latex,     
    	on grid=true,
    	scale=0.8,
    	every node/.style={scale=0.8}
    	]
		
	\begin{scope}[name prefix = txfund-]
		\TxBoxWithCondition{0}{-1.5}{$\lock{btc}$}{$a \land b$}
	\end{scope}
		
		\begin{scope}[name prefix = txredeem-]
			\TxBoxWithoutCondition{4}{0}{$\redeem$}{$\address{redeem}{A}$}
		\end{scope}
			
	\begin{scope}[name prefix = txcancel-]	
		\TxBoxWithCondition{5}{-3}{$\cancel$}{$a \land b$}
	\end{scope}		
	
\begin{scope}[name prefix = txrefundfull-]
	\TxBoxWithoutCondition{8}{1}{$tx_{\textsf{full refund}}^{\textsf{btc}}$}{$\address{refund}{B}$}
\end{scope}

\StickyNote[10em]{txrefundfull-txBox}{right=7.5em}{noteFullRefund}{Alice can (choose) to sign her part of this transaction and transmit the PSBT to Bob off-chain during any point before $\mathsf{btc\_partial\_refund}$ is published. Bob can then sign his part of the transaction and publish it to the network to claim a full refund. In comparison to ``partial refund + amnesty'', this saves fees but requires interactivity.}


\begin{scope}[name prefix = txtake-]
	\node[rectangle] at (8.5,-1.5) (center) {};
	\node[txLabel, above=1em of center] (nodeLabel) {$tx_{\textsf{partial refund}}^{\textsf{btc}}$};
	\node[rectangle, draw, minimum height=1.5em, minimum width=8.5em, below=0.5em of center] (firstBox) {};
	\node[font=\scriptsize, text centered, below=0.5em of center] (firstOutput) {$\address{refund}{B}~(90\%)$};
	\node[rectangle, draw, minimum height=1.5em, minimum width=8.5em, below=1.5em of firstOutput] (secondBox) {};
	\node[font=\scriptsize, text centered, below=1.5em of firstOutput] (secondOutput) {$a \land b~(10\%)$};
	\node[IF, right=2.4em of secondOutput] (spendingCondition) {};
	\node[rectangle, inner sep=0, minimum height=0, below=1em of secondBox] (bottomPadding) {};
	\node[rectangle, draw, rounded corners, minimum width=9.5em, fit=(nodeLabel)(firstBox)(secondBox)(spendingCondition)(bottomPadding)] (txBox) {};
\end{scope}

\StickyNote[13em]{txtake-txBox}{above right=2em and 9em}{notePartialRefund}{Alice signs her part of this transaction during the negotiation phase and transmits it to Bob. Bob will signs his part of this transaction once tx\_cancel is published on chain and Alice has not transmitted tx\_btc\_full\_refund to him until then. Alice can extract S\_B from the transaction once she sees it}
		
	\begin{scope}[name prefix = txamnesty-]
		\node[rectangle] at (12,-3) (center) {};
		\node[rectangle, text width=7em, text centered, rounded corners, minimum height=2em, minimum width=8em, above=1em of center] (nodeLabel) {$tx_{\textsf{refund amnesty}}^{\textsf{btc}}$};
		\node[txSpendingLabel, below=1em of center] (spendingLabel) {$\address{refund}{B}$};
		\node[rectangle, draw, minimum height=1.5em, minimum width=8em, below=1em of center] (labelBox) {};
		\node[rectangle, draw, rounded corners, minimum height=5em, minimum width=8.5em, below=-0.em of center] (txBox) {};
	\end{scope}

	\StickyNote[10em]{txamnesty-txBox}{below left=5em and 3em}{noteAmnesty}{If Bob has done a partial refund, Alice can (choose) to publish this transaction to allow him a full refund. Bob signs this part of the transaction during the negotiation phase. Alice can publish this non-interactively.}
	
	\begin{scope}[name prefix = txpunish-]
		\TxBoxWithoutCondition{8}{-4.5}{$\punish$}{$\address{punish}{A}$}
	\end{scope}	
							
	\draw [line] (txfund-spendingCondition.north) |- (txredeem-txBox.west) 
			node [labelBelow_small, font=\scriptsize]{$\A, \B$};
		
		\draw [line] (txfund-spendingCondition.south) |- (txcancel-txBox.west) 
		        node [labelAbove_small, font=\scriptsize] {$\checkRelative{t_1}$} 
				node [labelBelow_small,  font=\scriptsize]{$\A, \B$};
		
	\draw [line] (txcancel-spendingCondition.north) |- (txrefundfull-txBox.west)
		node [labelBelow_small,  font=\scriptsize]{$\A, \B$};

	\draw [line] (txcancel-spendingCondition.north) |- (txtake-txBox.west)
		node [labelBelow_small,  font=\scriptsize]{$\A, \B$};
	
	\draw [line] (txtake-spendingCondition) |- (txamnesty-txBox.west)
		node [labelBelow_small,  font=\scriptsize]{$\A, \B$};
					
	\draw [line] (txcancel-spendingCondition.south) |- (txpunish-txBox.west)
			node [labelAbove_small, font=\scriptsize] {$\checkRelative{t_2}$} 
			node [labelBelow_small,  font=\scriptsize]{$\A, \B$};
				
		\draw [-,dotted] (-1,-6) -- (9,-6);

		
		\begin{scope}[name prefix = xmr-lock-]
			\TxBoxWithCondition{0}{-7.5}{$\lock{xmr}$}{$\Spend{a}{} + \Spend{b}{}$}
		\end{scope}
		
		\begin{scope}[name prefix = xmr-redeem-]
			\TxBoxWithoutCondition{4}{-7.5}{$tx_{\textsf{redeem}}^{\textsf{xmr}}$}{$\textsf{Bob}$}
		\end{scope}
		
		\begin{scope}[name prefix = xmr-cancel-]
			\TxBoxWithoutCondition{4}{-9.5}{$tx_{\textsf{refund}}^{\textsf{xmr}}$}{$\textsf{Alice}$}
		\end{scope}
						
		\draw [dotted] (xmr-lock-spendingCondition.north) |- (xmr-redeem-txBox.west)
				node [labelBelow_small,  font=\scriptsize]{$\s{A}, \s{B}$};
												
		\draw [dotted] (xmr-lock-spendingCondition.south) |- (xmr-cancel-txBox.west)
				node [labelBelow_small,  font=\scriptsize]{$\s{A}, \s{B}$};
		
	\end{tikzpicture}
\caption{Transaction schema for BTC to XMR atomic swaps. \textit{Top}: Transaction schema for Bitcoin. \textit{Bottom}: Transaction schema for Monero. 
	\textit{Note: Monero view keys are omitted for clarity.}}
\label{fig:old_btc_protocol}
\end{figure}
